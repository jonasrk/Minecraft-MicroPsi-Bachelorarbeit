\chapter{Introduction}
\label{chap:1}

The hunt for artificial intelligence (AI) started many years ago. AI research is conducted for many purposes. A rough separation of these can be made, by separating the goals into developing useful, working applications and learning about the nature of intelligence itself. The ultimate task of latter---recreating human intelligence---admittedly still seems to be science fiction.

A new generation of cognitive scientists, psychologists and computer scientists strives to implement new ideas for simulated minds by building cognitive architectures, which implement AI systems that are inspired by the human brain. Many of them do so by simulating, in one way or the other, what are called neural node nets.~\cite{Goertzel201030} To test the functionality of such cognitive architectures and to figure out their capabilities and potential, we need to study their behaviour inside defined environments. As implementing AI into the physical world (as robots, for example) requires patience and building appropriate hardware, computer-simulated environments play an important role.

One of these architectures is MicroPsi 2~\cite{conf/agi/Bach12}. Developed by Joscha Bach and Dominik Welland, it is based on the Psi theory by Dietrich Dörner~\cite{Doerner98}, a German professor for theoretical psychology. It aims at providing a reliable and complete implementation of his theory of cognition while at the same time being easily accessible, understandable and modifiable for the research and applications to come. MicroPsi 2 is both: a cognitive architecture as well as an interface to a set of different simulation environments.

\section{Motivation}

Video games are natural applications of artificial intelligence. The quality of a game's AI can make the difference between an immersive experience and an uninspired demonstration of computer graphics technology. One game that especially stood out in the recent years is Minecraft. As a so called \emph{sandbox game}~\cite{Duncan:2011:MBC:2207096.2207097}, it attempts to leave the player with as few limitations as possible for what they do and how they behave in the game world~\cite{doi:10.1162/dmal.9780262693646.167}. This kind of open world delivers a simulated environment rich of opportunities, for both human and artificial players. 

Even though MicroPsi 2 already has integrated simulation environments, it is aimed for to experiment with offbeat ideas for AI simulations to generate new insights and opportunities for future research. Building an interface in between the cognitive architecture MicroPsi 2 and the video game Minecraft is what this thesis is about.

\section{Outline}
This project is fundamentally about combining existing technologies. The most important ones being MicroPsi 2, the framework aiming to implement the ideas of the Psi theory, and Minecraft, the popular sandbox videogame.

To understand how and why they were chosen, a brief history of their evolution as well as explanations of their basic ideas and relevant insights into their architecture are given in chapter \ref{chap:2} and \ref{chap:3}. Chapter two introduces Artificial General Intelligence and in particular the Psi theory and its implementations. We will focus on the particular implementation MicroPsi 2 and describe its modules in detail. Chapter \ref{chap:3} introduces the video game Minecraft, its protocol and related software projects. As the resulting software heavily relies on them, the protocol and the utilised open source software will get increased attention. 
In Chapter \ref{chap:4} the contribution of this thesis, the Minecraft world adapter for MicroPsi 2, is being described in detail. As a proof of concept, the description eventually leads to a small experiment.
Chapter \ref{chap:5} summarises the findings of this thesis and gives an outlook on possible future applications.