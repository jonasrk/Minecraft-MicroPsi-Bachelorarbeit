\chapter{Introduction}
% TODO Die Introduction fängt gut an, aber MicroPsi fällt mir zu sehr vom Himmel. Du solltest in der Introduction wirklich nur motivierend bleiben, warum will man so was wie MicroPsi haben, warum braucht man Simulationsumgebungen, keine Details zur MicroPsi-Implementierung oder Minecraft. Das kommt erst im Kapitel Foundations (dort kannst du den Text wahrscheinlich ziemlich so bringen, wie er in der Motivation steht)! 
The hunt for artificial intelligence started many years ago. Dividing the subjects into the strong and weak parts means seperating it goals into usefull applications and those that try to learn about the nature of intelligence itself. The ultimate task of strong A.I. --- recreating human intelligence --- admitably still seems to be science fiction, though.

But then again, a new generation of cognitive scientists, psychologists and computer scientists strives to implement new Ideas for simulated cognition --- building cognitive architectures. Many of them do so by simulating in one way or the other, what could be called "neural networks(/nodenets)".

To test the functionality of such cognitive architectures and to figure out their capabilities and potential we need to research their behavior inside defined environemts. As implementing AI into the physical world (as robots for examples) requires building apropriate hardware and patience, computer-simulated environments are play an important role.

MicroPsi is both: a cognitive architecture and also a set of and an interface to simulation environments.

\section{Motivation}
Video games are natural applications of artificial intelligence. The quality of a games A.I. can make all the difference in between a great title and an uninspired demo of computer graphics technology.

\section{Objective}

\section{Approach}

\section{Outcome}


\section{Outline}
...