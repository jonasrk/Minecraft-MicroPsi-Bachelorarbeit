\chapter{Introduction}
\label{chap:1}

%TODO Ist das eine offizielle Bezeichnung "strong and weak"? Klingt sehr negativ!
%TODO !! Hat dieses strong was mit den strong and weak Parts von vorher zu tun?

The hunt for artificial intelligence (AI) started many years ago. Dividing the subject into the strong and the weak part means separating its goals into developing useful, working applications and learning about the nature of intelligence itself. The ultimate task of strong AI---recreating human intelligence---admittedly still seems to be science fiction, though.

%TODO But then again? Was soll das heißen?
%TODO !! Was sind cognitive architectures in a nutshell?

But then again: a new generation of cognitive scientists, psychologists and computer scientists strives to implement new ideas for simulated cognition by building cognitive architectures. Many of them do so by simulating, in one way or the other, what we would call neural node nets.

One of these architectures is MicroPsi 2~\cite{conf/agi/Bach12}. Developed by Joscha Bach and Dominik Welland, it is based on the Psi theory by Dietrich Dörner~\cite{Doerner98}, who is a German professor for theoretical psychology. It aims at providing a reliable and complete implementation of his theory of cognition and at the same time being easily accessible, understandable and modifiable for the research and applications to come.

To test the functionality of such cognitive architectures and to figure out their capabilities and potential, we need to study their behaviour inside defined environments. As implementing AI into the physical world (as robots, for example) requires patience and building appropriate hardware, computer-simulated environments play an important role.

MicroPsi 2 is both: a cognitive architecture as well as a set of and an interface to simulation environments.

\section{Motivation}

%TODO Zitat für Sandbox Game und frew limitations as possible suchen

Video games are natural applications of artificial intelligence. The quality of a game's AI can make the difference between an immersive experience and an uninspired demonstration of computer graphics technology. One game that especially stood out in the recent years is Minecraft. As a so called \emph{sandbox game}, it attempts to leave the player with as few limitations as possible, for what they do and how they behave in the game world. This kind of relatively open world delivers a simulated environment rich of opportunities, for both human and artificial players. Building an interface in between the cognitive architecture MicroPsi 2 and the video game Minecraft is what this thesis is about.

%TODO !! Vorher hast du gesagt, dass MicroPsi bereits Simulationsumgebungen enthält. Wie passt das mit deiner Bachelorarbeit zusammen?

\section{Outline}
This project is fundamentally about combining existing technologies. The most important ones being MicroPsi 2, the framework aiming to implement the ideas of the Psi theory, and Minecraft, the popular sandbox videogame.

To understand how and why they were chosen, a brief history of their evolution as well as explanations of their basic ideas and relevant insights into their architecture are given in chapter \ref{chap:2} and \ref{chap:3}. Chapter two introduces Artificial General Intelligence and in particular the Psi theory and its implementations. We will especially focus on the particular implementation MicroPsi 2 and describe its modules in detail. Chapter \ref{chap:3} introduces the video game Minecraft, its protocol and related software projects. As the resulting software heavily relies on them, the protocol and the utilised open source software will get increased attention. 
In Chapter \ref{chap:4} the contribution of this thesis, the Minecraft world adapter for MicroPsi 2, is being described in detail. As a proof of concept, the description eventually leads to a small experiment.
Chapter \ref{chap:5} summarises the findings of this thesis and gives an outlook on possible future applications.