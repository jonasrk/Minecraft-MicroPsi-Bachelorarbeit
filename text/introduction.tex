\chapter{Introduction}
% TODO Die Introduction fängt gut an, aber MicroPsi fällt mir zu sehr vom Himmel. Du solltest in der Introduction wirklich nur motivierend bleiben, warum will man so was wie MicroPsi haben, warum braucht man Simulationsumgebungen, keine Details zur MicroPsi-Implementierung oder Minecraft. Das kommt erst im Kapitel Foundations (dort kannst du den Text wahrscheinlich ziemlich so bringen, wie er in der Motivation steht)! 
The hunt for artificial intelligence started many years ago. Dividing the subject into the strong and the weak part means separating its goals into useful applications and those that try to learn about the nature of intelligence itself. The ultimate task of strong A.I. --- recreating human intelligence --- admittedly still seems to be science fiction, though.

But then again: a new generation of cognitive scientists, psychologists and computer scientists strives to implement new ideas for simulated cognition, by building cognitive architectures. Many of them do so by simulating, in one way or the other, what could be called neural node nets.

One of these architectures is MicroPsi 2. Developed by Joscha Bach and Dominik Welland, it is based on the Psi Theory by Dietrich Dörner, who is a german Professor for theoretical psychology. It aims at providing a solid and complete implementation of his theory of cognition and at the same time being easily accessible, understandable and modifiable for the research and applications to come.

To test the functionality of such cognitive architectures and to figure out their capabilities and potential, we need to research their behaviour inside defined environments. As implementing AI into the physical world (as robots for examples) requires building appropriate hardware and patience, computer-simulated environments therefore play an important role.

MicroPsi 2 is both: a cognitive architecture but also a set of and an interface to simulation environments.

\section{Motivation}
Video games are natural applications of artificial intelligence. The quality of a games A.I. can make all the difference in between a great title and an uninspired demo of computer graphics technology. One game that especially stood out in the past is Minecraft. As a so called sandbox game, it attempts to give the player as few limitations as possible for what they do in the game world. This kind of relatively open environment, delivers a a simulated world rich of opportunities, to both human and artificial players. Building an interface in between the cognitive architecture MicroPsi 2 and the video game Minecraft is what this Thesis is about.


\section{Outline}
Chapter two and three give an overview of the necessary foundations for this project. Chapter two introduces the Psi theory and its implementations. Chapter three introduces the video game Minecraft, its protocol and related software projects.
In Chapter four the contributions of this thesis --- the Minecraft world adapter for MicroPsi --- are described in detail. As a proof of concept, the description eventually leads to a small experiment.
Chapter five summarises the findings of this thesis and gives an outlook on possible future applications.