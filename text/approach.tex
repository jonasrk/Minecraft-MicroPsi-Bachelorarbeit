\chapter{A! Approach / Minecraft as a Simulation Environment for MicroPsi 2}

The objective of this thesis is to build and test an interface in between MicroPsi and Minecraft, so that a Minecraft world (e.g. server) can be used as a simulation environment for the MicroPsi 2 Framework, which will act as an artificial player.

The modular architecture of MicroPsi 2 allows it to add new simulation environments (or worlds, as they are called in microPsi) fairly easy. A world needs interfaces to Data Source and Data Targets and a step-function that evolves the world.

On the other hand, communication with a Minecraft Server typically requires a constant flow of data packets going in and out. Most third party clients, including Bots, facilitate own event loops. What has been done for this project, was breaking down the event loop of the used bot framework and rebuild it as the step function of the MicroPsi 2 world. It should be noted, that the frequency in which the framework steps the bot has to be at least chosen so, that it is able to send keep-alive-signals to Server, to not get kicked.

That being said, a big part of the project is about visualization. Inside the MicroPsi Core Application, a 3D-visualization of the Minecraft world and the agent within is aimed for. There are two main reasons for this goal. The first reason is, that the agents behavior within the simulation environment is supposed to be monitored from the MicroPsi webinterface in an aesthetically appealing way. The secons reason is, that the image data is supposed to be processed by the agent as one of it's senses.

Do obtain these goals, the Minecraft Client-Server-Protocol had to be researched, learned and imitated.

Then, artificial Minecraft players (written in Python) had to be searched, found and researched.

Eventually, building upon an existing Bot framework led to an integration with the MicroPsi framework.

\section{A! Overview / What has been there so far?}
... Minecraft Bots with simple as well as sophisticated AI ...
... MicroPsi 2 with Island and Berlin world ...

\section{A! Architecture / Building the interface in between Minecraft and the simulation environment}
... result: a Minecraft Bot that implements MicroPsi AI and is controlled and monitored via the MicroPsi webinterface ...
... the Webinterface holds its own visualization of the Agents worldview ...

\subsection{A Minecraft Bot}

\subsubsection{Protocol Implementation}

\subsubsection{Control Structures}

\subsubsection{Previous own implementations with TwistedBot}

\subsubsection{other popular Bot projects and game modifications}

\subsubsection{Spockbot von Nickelpro}

\subsection{Implementing the Bot in MicroPsi}

\subsection{The MicroPsi side}

\subsection{necessary Modifications and Additons in core/worldrunner}

\paragraph{Data Targets and sources}

\subsection{necessary Modifications and Additons in server/control and monitoring interface}

\subsection{The Visualization}

\subsubsection{Requirements and necessity of a visualization}
... why a visiualization and what is it supposed to do? ...

\paragraph{monitoring the bot from the webinterface}
... make it aesthetically appealing as well as easily accessible ...

\paragraph{using visualization output as a Datasource / as the bots eyes}

\subsubsection{Implementation}

\paragraph{used Data}
... required Data for the visualization ...
... and how to obtain it (first attempts: telnets/then sockets) ...

\subsubsection{3D Visualisierung mit Pyglet}
... foundation: that minecraft pyglet clone ...

\subsubsection{Earlier attempts using JavaScript / AJAX}
... worked well but a little slow ...

\section{A! Implementation}
% TODO je nach Größe des Kapitels, das hier vielleicht eine Ebene höher ziehen (versuche dich hier erstmal nur auf die fertige Lösung zu konzentrieren und warum du dich dafür entschieden hast. Es ist nicht interessant, was du sonst alles ausprobiert hast, außer dass du konkret angibst, warum du dich für die jetztige Implementierung im Vergleich zu anderen entschieden hast)

\section{A! Case Study}

The functionality is to be tested with a simple Braitenberg-vehicle experiment.

\subsection{Experiment}
... experiment to test functionality of the system ...
... scope: only a simple test for time reasons ...

\subsubsection{Braitenberg Vehicle}
... simplest proof of concept of a microPsi agent ...