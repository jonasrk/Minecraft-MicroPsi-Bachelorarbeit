\chapter*{\centering \begin{normalsize}Zusammenfassung\end{normalsize}}
\begin{quote}
Ziel der Arbeit ist die Entwicklung und das Testen einer Simulationsumgebung für einen kognitiven Agenten auf Basis des populären Videospiels Minecraft.

Minecraft bietet sich als Grundlage für eine Simulationsumgebung aufgrund der kompositionalen Semantik der Spielwelt besonders an, da das Agentensystem durch das Erforschen seiner Umwelt Wissen über die Spielwelt aufbauen und ähnliche Strukturen wieder erkennen kann. Objekte der Spielwelt sind in Minecraft keine bloßen Hindernisse sondern werden mit variablen Eigenschaften prozedural erzeugt und ähneln so eher einer realen Umgebung als andere virtuelle Welten.

Da Minecraft von Anfang an mit einem Mehrspielermodus ausgestattet wurde, lässt es sich zudem für Multiagenten-Umgebungen und so für kollaborative Agenten verwenden. Des Weiteren sind Minecraft-Lizenzen günstig zu erwerben, erhältlich für viele Plattformen und es gibt eine äußerst große und aktive Community für selbsterstellte Spiel-Inhalte und -Modifikationen.

Angestrebt wird, dass die kognitive Architektur MicroPsi2 sich an einen Minecraft-Server anmelden kann, ihre Umgebung wenigstens in Ansätzen wahrnimmt (Objekte, Terraintypen), sich fortbewegen kann und einfache Interaktionsmöglichkeiten besitzt (z.B. Objekt aufnehmen und ablegen). 

Daneben soll eine Visualisierung der Umgebung aus Sicht des Agenten entstehen, z.B. als zweidimensionale Übersicht, in deren Zentrum der Agent steht. Diese Visualisierung soll live im Browser angezeigt werden (wahlweise mit WebGL oder Canvas/D3).

Zum Schluss der Arbeit soll zum Testen der Funktionalität ein virtuelles Braitenbergvehikel in die Simulationsumgebung gesetzt werden, welches sich daraufhin auf die nächstgelegene Lichtquelle zubewegen soll. Dieses Experiment wird als Teil der Arbeit umfangreich dokumentiert.\end{quote}

\chapter*{\centering \begin{normalsize}Abstract\end{normalsize}}
\begin{quote}
  What I cannot create, I do not understand
  Richard P. Feynman, 1988
\end{quote}