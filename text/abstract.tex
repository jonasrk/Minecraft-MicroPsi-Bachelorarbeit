\chapter*{\centering \begin{normalsize}Zusammenfassung\end{normalsize}}
\begin{quote}


Simulationsumgebungen spielen für das Testen und Erforschen von neuen Ansätzen für künstliche Intelligenz eine entscheidende Rolle. Da sich Computerprogramme nur mit erhöhtem technischen und finanziellen Aufwand in die physische Welt implementieren lassen, können simulierte Umgebungen schnellere, günstigere und reproduzierbarere Ergebnisse liefern. Genauso wie Innovationen aus dem Bereich der KI neue Impulse setzen, können auch neue Ansätze für Simulationsumgebungen zu neuen Erkenntnissen führen.

Das populäre Videospiel Minecraft bietet sich durch die kompositionale Semantik der Spielwelten besonders als Grundlage für eine Simulationsumgebung an, da das Agentensystem durch das Erforschen seiner Umwelt, Wissen über die Spielwelt aufbauen und sich ähnelnde Strukturen wiedererkennen kann. Objekte der Spielwelt sind in Minecraft keine bloßen Hindernisse, sondern werden mit variablen Eigenschaften prozedural generiert. Durch die Komposition dieser Objekte entstehen Umgebungen, die einer realen Umgebung in dieser Hinsicht eher gleichen, als andere virtuelle Welten. Da Minecraft einen umfangreichen Mehrspielermodus beinhaltet, lässt es sich zudem für Multiagenten-Umgebungen und so für Simulationen mit kollaborativen Agenten verwenden. Darüber hinaus sind Minecraft-Lizenzen günstig zu erwerben, erhältlich für viele Plattformen und es gibt eine äußerst große und aktive Community für selbsterstellte Spiel-Inhalte und Modifikationen.

Diese Arbeit präsentiert eine Schnittstelle, welche es der kognitiven Architektur MicroPsi 2 ermöglicht, sich an einem Minecraft Server anzumelden, die Umgebung wahrzunehmen und sich darin fortzubewegen. Im Anschluss daran wird die Visualisierungskomponente vorgestellt, welche eine mit Open-GL gerenderte 3D-Ansicht der Simulationsumgebung live im MicroPsi 2 User Interface anzeigt. Es folgt die Beschreibung eines einfachen Experimentes, bei welchem sich ein Agent selbstständig auf ein vorher definiertes Objekt zubewegen muss. Die Arbeit endet mit der Dokumentation und Auswertung des Experimentes.\end{quote}

\chapter*{\centering \begin{normalsize}Abstract\end{normalsize}}

\begin{quote}

Simulation environments play an important role for testing and researching new approaches to artificial intelligence. Because computer software can only be implemented into the physical world with increased technical and financial effort, simulated environments can deliver results faster, cheaper and more reproducible. In the same way as innovations in AI can deliver new impulses, new approaches to simulation environments can just as well lead to new insights into the future of intelligent machines.

Because of its compositional semantics, the popular video game Minecraft turns out to be an interesting simulation environment, as agents can generate knowledge through exploring their environment and recognising similar structures. Objects in Minecraft worlds are not just obstacles, but are generated procedurally with varying characteristics. Through composition of these objects, environments emerge that in this aspect are more similar to real environments than other virtual worlds. Since Minecraft contains an extensive multiplayer mode, it is also suitable for multi-agent environments and for simulations with collaborative agents. Furthermore, Minecraft licenses are affordable and available for many platforms and there is a huge and active community for player generated content and modifications.

This thesis presents an interface which enables the cognitive architecture MicroPsi 2 to connect to a Minecraft server, perceive its environment and move around within it. Subsequently, the visualisation component is introduced that displays an OpenGL rendered 3D view live in the Micro Psi 2 user interface. It is followed by the description of a simple experiment in which the agent has to move towards a previously specified object. The thesis concludes with the documentation and evaluation of the experiment.
\end{quote}