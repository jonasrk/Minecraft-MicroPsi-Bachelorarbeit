\chapter*{\centering \begin{normalsize}Zusammenfassung\end{normalsize}}
\begin{quote}

%TODO  Viele Zeitsprünge: entscheide dich für eine
%TODO Der Abstrakt springt etwas zwischen Motivation und Erklärung hin zu einer Aufzählung was du gemacht hast. Er wirkt dadurch etwas holprig.

Simulationsumgebungen spielen für das Testen und Erforschen von neuen Ansätzen für künstliche Intelligenz eine entscheidende Rolle. Da sich Computerprogramme nur mit erhöhtem technischem und finanziellem Aufwand in die physische Welt implementieren lassen, können simulierte Umgebungen schnellere, günstigere und reproduzierbarere Ergebnisse liefern. Genauso wie Innovationen aus dem Bereich der KI neue Impulse setzen, können auch neue Ansätze für Simulationsumgebungen zu neuen Erkenntnissen führen.

%TODO erzeugen hierdurch Welten?

Diese Arbeit präsentiert ein Schnittstelle, welche es der kognitiven Architektur MicroPsi 2 ermöglicht, Agenten in Umgebungen des populären Videospiels Minecraft zu steuern. Minecraft bietet sich als Grundlage für eine Simulationsumgebung aufgrund der kompositionalen Semantik der Spielwelt besonders an, da das Agentensystem, durch das Erforschen seiner Umwelt, Wissen über die Spielwelt aufbauen und ähnliche Strukturen wiedererkennen kann. Objekte der Spielwelt sind in Minecraft keine bloßen Hindernisse, sondern werden mit variablen Eigenschaften prozedural erzeugt und erzeugen hierdurch Welten, die einer realen Umgebung mehr ähneln als andere virtuelle Welten.

%TODO Silbentrennung einführen, entweder direkt hier oder mit /hyphenation

Da Minecraft von Anfang an mit einem Mehrspielermodus ausgestattet wurde, lässt es sich zudem für Multiagenten-Umgebungen und so für kollaborative Agenten verwenden. Darüber hinaus sind Minecraft-Lizenzen günstig zu erwerben, erhältlich für viele Plattformen und es gibt eine äußerst große und aktive Community für selbsterstellte Spiel-Inhalte und -Modifikationen.

Die für diese Arbeit entwickelte Schnittstelle ermöglicht es MicroPsi, sich an einem Minecraft Server anzumelden, die Umgebung wahrzunehmen und sich fortzubewegen. Darauf aufbauend wurde eine Visualisierung der Umgebung aus Sicht des Agenten implementiert, welche als OpenGL gerenderte 3D-Ansicht live im Browser angezeigt wird.  Zum Schluss der Arbeit wird ein einfaches Experiment durchgeführt, bei welchem sich der Agent selbstständig auf ein bestimmtes Objekt zubewegen muss. Dieses Experiment wurde als Teil der Arbeit umfangreich dokumentiert.\end{quote}

\chapter*{\centering \begin{normalsize}Abstract\end{normalsize}}

\begin{quote}

%TODO Nicht nur ein Satz pro Absatz

Simulation environments play an important role for testing and researching new approaches to artificial intelligences. Because computer software can only be implemented into the physical world with increased technical and financial effort, simulated environments can deliver results faster, cheaper and more reproducible.

In the same way as innovations in AI can deliver new impulses, new approaches to simulation environments can just as well lead to new insights into the future of intelligent machines..

This thesis presents an interface, which enables the cognitive architecture MicroPsi 2, to control agents in environments of the popular video game Minecraft.

Because of its compositional semantic, Minecraft turns out to be an interesting simulation environment, as agents can generate knowledge through exploring their environment and may recognise similar structures. Objects in Minecraft worlds are not just obstacles, but are generated procedurally with varying characteristics and therefore make the worlds more similar to real environments than other virtual worlds.

Since Minecraft contained a multi player mode from its early days on, it is also suitable for multi agent environments and therefore for collaborative agents. Furthermore, Minecraft licenses are affordable and available for many platforms. Also, there is a huge and active community for player generated content and modifications.

The interface enables MicroPsi, to connect to a Minecraft server, perceive its environment and move around within it.

Additionally, a visualisation of the environment, as it is perceived by the agent, has been implemented. It displays an OpenGL rendered 3D view live in the web browser.

Finally, a simple experiment is concluded, in which the agent has to move towards a specified object. This experiment is documented as a part of this thesis.
\end{quote}