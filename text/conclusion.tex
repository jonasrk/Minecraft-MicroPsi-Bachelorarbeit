\chapter{Conclusion}
\label{chap:5}
This thesis presented an implementation of an interface that makes it possible to use servers of the video game Minecraft as a simulation environment for the cognitive artificial intelligence framework MicroPsi 2. After several iterations and trying out different approaches and technologies, the interface is now functional. The proof-of-concept experiment with an agent seeking a particular block resulted in proofing that Minecraft is usable as a simulation environment.

In particular, an artificial player may now connect to a Minecraft world and be controlled by a MicroPsi node net. Because there is a wide variety of possible input data from Minecraft worlds and of possible actions artificial players can perform, corresponding data sources and targets may furthermore be defined freely inside the world adapter code.

The interface is in its core based on two open source software projects. The Python Minecraft bot framework \texttt{Spock} serves as the client that connects to Minecraft servers. The Python Minecraft clone \texttt{Minecraft} by Michael Fogleman serves as the foundation for the visualisation component.

The implemented case study shows that the interface as a whole works as expected. Because it so far serves foremost as a proof-of-concept, there is a lot of room left for improvement. The two main critical aspects are speed and robustness. Speeding the simulation up would require more profiling and finding bottle necks in the data flow. Most likely, at this time one of the major bottlenecks will be, that, for every iteration of the visualisation component, an image file is written to the hard drive.

The Minecraft protocol proved to be easily accessible, understandable and therefore gratifying to work with. It sets the hurdles for programming against Minecraft very low, so there might even be research projects with other scopes than AI simulation that could make use of Minecraft infrastructures.

Possible future applications of this project would be implementing data sources and targets for every aspect of a world and for every Minecraft action that one wants to experiment with. For example, these could include interacting with the environment by picking up objects, creating and destroying blocks, building, mining, interacting and fighting with other objects. Especially, in the future, multiple agents shall interact within the same environment and collaborate with each other.

A complex experiment that would make use of the unique capabilities of MicroPsi and Minecraft could look like the following:
A group of Psi agent get placed in a Minecraft world. They are equipped with a set of predefined drives. These could in the simplest be the need to consume specific resources like water and food at least once a day. As they would not know about these drives themselves, they would start exploring their environment randomly. Soon they would learn, what actions lead to pleasure and what to displeasure signals. Additionally they would start learning the semantics of the game world to recognise related objects, even though they look different. Ideally, the agents would communicate their knowledge in between each other and therefore help each other to survive longer. In an identical world human players could be given the same task. Ultimately, their behaviours could be compared to find out, if the Psi theory is implement to recreate human cognition.