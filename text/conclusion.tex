\chapter{Conclusion}
%Analyse und Zusammenfassung / Auswertung der Ergebnisse / Diskussion der Ergebnisse und weitere Erkenntnisse / Limitationen und Ausblick
After several iterations and trying out different approaches and technologies the Interface is now functional.

The experiment with the simulated Braitenbergvehikel resulted in proofing that Minecraft is usable as a simulation environment.

This Thesis presented an implementation of an interface, that makes it possible to use servers of the video game Minecraft as a simulation environment for the cognitive artificial intelligence framework MicroPsi 2. In particular, an artificial player may no connect to a Minecraft world and be controlled by a MicroPsi node net. Because there is a wide variety of possible input data from Minecraft world and of possible actions artificial players can perform, corresponding data sources and targets may be defined freely at a defined place.

The interface is in its core based on two open source software projects. The Python Minecraft bot framework Spock serves as the client that connects to Minecraft Servers. The Python Minecraft Clone Minecraft by Michael Fogleman serves as the foundation for the visualisation component.

The implemented case study shows, that the interface as a whole functions as expected. Although, it so far servers foremost as a proof-of-concept, an there is a lot of room for improvement, with the two main flaws being speed and and robustness. Speeding the simulation up would require profiling and finding the bottle necks in the data flow. Most likely, at this time one of the major bottlenecks will be, that for every iteration of the visualisation component, an image file is written to the hard drive.

The Minecraft protocol proved to be easily accesible, understandable and therefore good to work with. It set's the barrier for programming against Minecraft very low, so there might even be research projects with other scopes than AI simulation, that could probably make use of Minecraft infrastructures.

\section{What's next}
Possible future applications of this project include implementing data targets for every Minecraft action that one wants to experiment with. For example, these could be interacting with the environment by picking up objects, creating and destroying blocks, building, mining, interacting and fighting with other objects. Espacially, in the future, multiple agents shall interact with the same environment and collaborate with each other.