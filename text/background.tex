\chapter{Background}
%TODO Die Introduction fängt gut an, aber MicroPsi fällt mir zu sehr vom Himmel. Du solltest in der Introduction wirklich nur motivierend bleiben, warum will man so was wie MicroPsi haben, warum braucht man Simulationsumgebungen, keine Details zur MicroPsi-Implementierung oder Minecraft. Das kommt erst im Kapitel Foundations (dort kannst du den Text wahrscheinlich ziemlich so bringen, wie er in der Motivation steht)!

A widely accepted definition of the field of A.I. is that it is the study and design of intelligent agents, where an intelligent agent is a system that perceives its environment and takes actions that maximize its chances of success. %TODO find original sources

It took many years for A.I. research to evolve from the early ideas of thinking machines over Deep Blue, the computer that could beat mankind's best chess players to Watson, the A.I. that beats the champions of Jeopardy, the game show, that is about asking the appropriate question to a given answer. There exist many applications for A.I.~. Self-driving cars and online-shopping recommendation systems to name a few.

These examples have one thing in common. They are applications of technology that serve an immediate, or at least foreseeable purpose. For A.I. in scenarios of this kind, the term weak A.I. (or applied A.I.) has been coined.

Strong A.I., in contrast, is about researching the nature of intelligence itself. An actual (hypothetical) implementation of a Strong A.I. translates to building a machine, that is capable of acting like a human being --- not just in a defined problem fields, but in all of them.

Cognitive A.I.~, in particular, can be thought of as architectures that implement findings and theories in cognitive science and psychology, as well as the neuro-sciences, for the sake of proving, if the theories hold against what they promise. 

%... Examples for cognitive architectures are ...

Many cognitive architectures share characteristics with or directly implement artificial neural nodenets.

%... Approaches to cognitive AI ...
%... neural node nets ...
%... related work ...

\section{Psi Theory}
The Psi theory in its foundations was described by german psychologist Dietrich Dörner in his book ``Bauplan für eine Seele'' and ``Die Mechanik des Seelenwagens'' from 1998 and 2002. It's main ideas are, that it thinks of cognition as a graph-like structure (e.g. node-net) of relationships that strives to maintain homeostatic balance.\cite{Bach:2009:PSI:1611304}

Basic components of the theory are Representation, Memory, Perception, Drives, Cognitive modulation and emotion, Motivation, Learning and Problem solving as well as Language and consciousness. %(wikipedia)

%TODO describe it myself
``The PSI theory is a theory of human action regulation by psychologist Dietrich Doerner. It is an attempt to create a body-mind link for virtual agents. It aims at the integration of cognitive processes, emotions and motivation. This theory is unusual in cognitive psychology in that emotions are not explicitly defined but emerge from modulation of perception, action-selection, planning and memory access. Emotions become apparent when the agents interact with the environment and display expressive behavior, resulting in a configuration that resembles emotional episodes in biological agents. The agents react to the environment by forming memories, expectations and immediate evaluations. This short presentation is a good overview.

PSI agents possess a number of modulators and built-in motivators that lie within a range of intensities. These parameters combined to produce complex behaviour that can be interpreted as being emotional. Additionally, by regulating these parameters, the agent is able to adapt itself to the different circumstances in the environment. This theory has been applied to different virtual agent simulations in different types of environments and has proven to be a promising theory for creation of biologically plausible agents.'' %(wikipedia)


%... Basics of Psi Theorie of Dörner ...

Joscha Bach adapted that theory to bring it in a contemporary from with slight modifications.

%... explanation of Joschas Dissertation ...

%\section{Summary}
Even though building a conscious machine that thinks and acts like we do is still mehre science-fiction, it is this kind of foundational research, that leads no new ways of thinking of the world, that give us our most import leaps.

%... still a lot to do in AI ...