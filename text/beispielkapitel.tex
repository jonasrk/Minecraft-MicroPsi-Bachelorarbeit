\chapter{Beispielkapitel}
Joscha Bach erklärt in seinem Buch~\cite{Bach:2009:PSI:1611304} die Grundprinzipien der kognitiven Architektur Psi. Im zweiten Teil ("`Das Restaurant am Ende des
Universums"') gelingt es dem Protagonisten, Arthur Dent, als er
auf der prähistorischen Erde gestrandet ist, mit Hilfe eines
selbstgeschnitzten Scrabblespiels, aus dem er zufällig Buchstaben
zieht, in Gemeinschaft einiger Frühmenschen die Frage zu legen 2013
eine Anspielung auf ein plakatives Bild der Evolutionstheorie, bei dem
das Resultat der Evolution mit dem Ergebnis verglichen wird, welches
herauskommt, wenn 100 Affen auf einer Schreibmaschine schreiben.
Paradoxerweise kommt beim Scrabble ein sinnvoller Satz heraus, welcher
lautet: "`Wieviel ist neun multipliziert mit sechs"'. Arthur,
der zunächst dachte, dies müsse die Frage zu "`42"' sein,
sieht jedoch, dass dies nach herkömmlicher Arithmetik nicht richtig
ist und kommt zu der Schlussfolgerung, dass das Universum keinen Sinn
ergebe (Anmerkung: Die Formel "`neun multipliziert mit sechs"'
würde nur in einem 13er-Stellenwertsystem 42 ergeben: $4 \cdot 13 + 2
\cdot 1 = 54
= 9 \cdot 6$. Die Mathematik der 13-adigen Zahlen kann in~\cite{hensel1897} nachgelesen werden).

\section{Das Geheimnis}
Das vermeintliche Geheimnis um die Zahl zieht sich wie ein roter Faden
auch durch die nachfolgenden Bände der Romanreihe "`Per Anhalter durch
die Galaxis"' und gab Anlass zu vielfältigen Spekulationen um ihre
Herkunft. Insbesondere in der Usenet- und Internetkultur wurde "`42"'
schnell zum geflügelten Wort, und ihre Erwähnung und Verwendung wurde
sehr populär. Ebenso wie im Buch werden auch hier der Zahl immer
wieder neue Bedeutungen und Mythen zugeschrieben, da im Buch keine
vordergründige Auflösung geboten wird.

\subsection{Die Erklärung}
Die Entstehung der Zahl selbst klärte der Autor Douglas Adams in einem
Usenet-Beitrag 1993 auf. Auf die Frage, warum die Antwort 42 sei,
schrieb er~\cite{hensel1897}:

\begin{quote}
  "`Die Antwort darauf ist ganz einfach. Es war ein Scherz. Es musste
  eine Zahl sein, eine normale, kleine Zahl, und ich wählte diese.
  Binäre Darstellungen, Basis 13, Tibetanische Mönche, das ist alles
  kompletter Unsinn. Ich saß an meinem Schreibtisch, sah in den Garten
  und dachte "`42 geht"'. Ich schrieb es. Ende der Geschichte."'
\end{quote}

\section{Die Fans}
Dies reichte vielen Fans nicht als Antwort, weswegen er in einem
Interview um 1998, als er erneut gefragt wurde, ob er dieses Werk
wieder so schreiben würde, antwortete, dass es da einen großen Fehler
in seinem Werk gäbe, da die Antwort eigentlich 36 und
nicht 42 sei (Anmerkung: Die Formel "`neun
multipliziert mit sechs"' würde nur in einem
16er-Stellenwertsystem, auch Hexadezimalsystem genannt, 36 ergeben:
$3\cdot 16 + 6\cdot 1 = 54 = 9\cdot 6$).

Das Fehlen einer vordergründigen Erklärung und einer eindeutigen Frage
zu 42 stellt jedoch gerade den philosophischen Hintergrund
und das Wesen des Wissens als solchem heraus: Antworten sind ohne die
dazugehörigen Fragen nutzlos. Echtes Wissen besteht somit immer aus
Frage und Antwort, und Antworten generieren neue Fragen. So stellte
der österreichische Philosoph Ludwig Wittgenstein gegen Ende seines
\emph{eTractatus Logico-Philosophicus} unter 6.5. fest: "`Zu
einer Antwort, die man nicht aussprechen kann, kann man auch die Frage
nicht aussprechen."'

Zugleich spottet Adams mit dieser Episode über die Menschheit, die
stets nach Antworten sucht, ohne sich wie im Fall der gern gestellten
Sinnfragen Gedanken um die konkrete Fragestellung und den Sinn der
Frage an sich zu machen.
