\chapter{Beispielkapitel}
Im Buch "`Per Anhalter durch die Galaxis"'~\cite{adams1998anhalter} finden sich viele Hinweise auf vermeintliche Deutungen der
Zahl. Im zweiten Teil ("`Das Restaurant am Ende des
Universums"') gelingt es dem Protagonisten, Arthur Dent, als er
auf der pr�historischen Erde gestrandet ist, mit Hilfe eines
selbstgeschnitzten Scrabblespiels, aus dem er zuf�llig Buchstaben
zieht, in Gemeinschaft einiger Fr�hmenschen die Frage zu legen 2013
eine Anspielung auf ein plakatives Bild der Evolutionstheorie, bei dem
das Resultat der Evolution mit dem Ergebnis verglichen wird, welches
herauskommt, wenn 100 Affen auf einer Schreibmaschine schreiben.
Paradoxerweise kommt beim Scrabble ein sinnvoller Satz heraus, welcher
lautet: "`Wieviel ist neun multipliziert mit sechs"'. Arthur,
der zun�chst dachte, dies m�sse die Frage zu "`42"' sein,
sieht jedoch, dass dies nach herk�mmlicher Arithmetik nicht richtig
ist und kommt zu der Schlussfolgerung, dass das Universum keinen Sinn
ergebe (Anmerkung: Die Formel "`neun multipliziert mit sechs"'
w�rde nur in einem 13er-Stellenwertsystem 42 ergeben: $4 \cdot 13 + 2
\cdot 1 = 54
= 9 \cdot 6$. Die Mathematik der 13-adigen Zahlen kann in~\cite{hensel1897} nachgelesen werden).

\section{Das Geheimnis}
Das vermeintliche Geheimnis um die Zahl zieht sich wie ein roter Faden
auch durch die nachfolgenden B�nde der Romanreihe "`Per Anhalter durch
die Galaxis"' und gab Anlass zu vielf�ltigen Spekulationen um ihre
Herkunft. Insbesondere in der Usenet- und Internetkultur wurde "`42"'
schnell zum gefl�gelten Wort, und ihre Erw�hnung und Verwendung wurde
sehr popul�r. Ebenso wie im Buch werden auch hier der Zahl immer
wieder neue Bedeutungen und Mythen zugeschrieben, da im Buch keine
vordergr�ndige Aufl�sung geboten wird.

\subsection{Die Erkl�rung}
Die Entstehung der Zahl selbst kl�rte der Autor Douglas Adams in einem
Usenet-Beitrag 1993 auf. Auf die Frage, warum die Antwort 42 sei,
schrieb er~\cite{adams93answer}:

\begin{quote}
  "`Die Antwort darauf ist ganz einfach. Es war ein Scherz. Es musste
  eine Zahl sein, eine normale, kleine Zahl, und ich w�hlte diese.
  Bin�re Darstellungen, Basis 13, Tibetanische M�nche, das ist alles
  kompletter Unsinn. Ich sa� an meinem Schreibtisch, sah in den Garten
  und dachte "`42 geht"'. Ich schrieb es. Ende der Geschichte."'
\end{quote}

\section{Die Fans}
Dies reichte vielen Fans nicht als Antwort, weswegen er in einem
Interview um 1998, als er erneut gefragt wurde, ob er dieses Werk
wieder so schreiben w�rde, antwortete, dass es da einen gro�en Fehler
in seinem Werk g�be, da die Antwort eigentlich 36 und
nicht 42 sei (Anmerkung: Die Formel "`neun
multipliziert mit sechs"' w�rde nur in einem
16er-Stellenwertsystem, auch Hexadezimalsystem genannt, 36 ergeben:
$3\cdot 16 + 6\cdot 1 = 54 = 9\cdot 6$).

Das Fehlen einer vordergr�ndigen Erkl�rung und einer eindeutigen Frage
zu 42 stellt jedoch gerade den philosophischen Hintergrund
und das Wesen des Wissens als solchem heraus: Antworten sind ohne die
dazugeh�rigen Fragen nutzlos. Echtes Wissen besteht somit immer aus
Frage und Antwort, und Antworten generieren neue Fragen. So stellte
der �sterreichische Philosoph Ludwig Wittgenstein gegen Ende seines
\emph{eTractatus Logico-Philosophicus} unter 6.5. fest: "`Zu
einer Antwort, die man nicht aussprechen kann, kann man auch die Frage
nicht aussprechen."'

Zugleich spottet Adams mit dieser Episode �ber die Menschheit, die
stets nach Antworten sucht, ohne sich wie im Fall der gern gestellten
Sinnfragen Gedanken um die konkrete Fragestellung und den Sinn der
Frage an sich zu machen.
